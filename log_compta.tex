\documentclass[12pt,a4paper,twoside]{book}
\usepackage[utf8]{inputenc}
\usepackage[francais]{babel}
\usepackage[T1]{fontenc}
\usepackage{amsmath}
\usepackage{amsfonts}
\usepackage{amssymb}
\usepackage{makeidx}
\usepackage{graphicx}
\usepackage{lmodern}


\author{Corentin LECLERE}
\title{ma compta}
\begin{document}
\chapter{...c'est la façon de s'en servir}

Dans la vie réelle, le contenu des comptes ne dépend pas de leur intitulé mais de la manière de s'en servir. Pour le prouver, je vais confronter la manière scolaire d'obtenir les informations nécessaires à calculer la tva dûe au titre d'un mois donné à la méthode qui m'a été décrite par mon maître de stage. 
Dans les deux méthodes, la tva dûe au titre du mois M, à payer le 21/M+1 se calcule ainsi: 
%Choisir entre déclarer sa tva simplement et se simplifier %la vie à longueur de mois

%Tous les mois, la tva dûe à l'état doit être déclarée. Elle vaut en première approche:
%\[due = tva(ventes) - tva(achats) \]
%
%Un cycle d'achat se présentera ainsi:


\begin{tabular}{cc}
   +&TVA collectée \\
  -&TVA déductible \\
  \hline
  =&TVA due.
\end{tabular}


La tva est collectée au titre du mois M sur les ventes de biens facturés et les ventes de services encaissées durant ce mois. 

Voyons comment se déroule un cycle de vente avec chèques, cartes bancaires et lcr avec la méthode canonique, et avec celle qui m'a été enseignée lors de mon stage, puis voyons l'évolution du contenu de la balance à m, m+1 et m+2. Quelles sont les conséquences sur le calcul de la TVA? Quelles mesures faut-t'il donc prendre pour calculer la tva avec la méthode "professionnelle"? 


avec la méthode canonique, on aura: 


\begin{tabular}{ccl}
+&44571&TVA collectée sur les ventes de biens et de services\\
-&44562&TVA déductible sur les achats d'immobilisations\\
-&44566&TVA déductible sur les achats de biens\\
\hline
=&&TVA due\\
\end{tabular}
 c pa gagné
 
\end{document}

